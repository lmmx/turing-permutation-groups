\documentclass[12pt]{article}
\usepackage[utf8]{inputenc}     % Ensure input is interpreted as UTF-8
\usepackage[T1]{fontenc}        % Good PDF font encoding
\usepackage{lmodern}            % Latin Modern fonts
\usepackage{amsmath,amssymb}    % Common math packages
\usepackage{hyperref}           % Optional: for hyperlinks (if needed)

\title{On Permutation Groups}
\author{Alan Turing}
\date{} % You can set a date or leave it empty

\begin{document}

\maketitle

\section{3}

Below is a careful, line-by-line transcription of the visible typed text and the marginal/handwritten edits in the image. 
Where something is struck out or inserted by hand, bracketed notes indicate it. 
Likewise, small bits of Turing's algebraic notation are reproduced (with unavoidable uncertainty in places). 
Ellipses ``...'' indicate letters or symbols too unclear to recover exactly.

\bigskip

Technique for investigating [handwritten above: ``any''] particular upright \(U\).

In order to prove \(H\) unexceptional it will suffice to prove
that \(J\) contains all three-cycles, for if this is so \(J\) will
be a self conjugate subgroup of \(S\), and since it is not the
identity it must be either \(A\) or \(S\). It would also be sufficient
to prove that \(J\) contains all 2-cycles. We shall prove [typed words struck out]
[handwritten ``Thereom 1'' or ``Theorem 1''] If \(J\) contains a member of form \((\alpha, R^m \alpha)\)
or \((\alpha, R^m \alpha)(\beta, \gamma)\) where \(m\) is prime to \(T\), then it
contains all three-cycles, and in the first case mentioned
all two-cycles. [handwritten note near this: ``must be greater than 4.
(\(\alpha, R^m \alpha\))(\(\beta,\gamma\)) must ... comm ...'']

\bigskip

Suppose \(J\) contains \((\alpha, R^m \alpha)\). We will write \(\alpha_k\)
for \(R^{m k}(\alpha)\). The symbols \(\alpha_0, \alpha_1, \ldots, \alpha_{r-1}\) include
all the \(T\) symbols. Then \(J\) contains
[handwritten: ``\(R^{(\alpha_0, \alpha_1)}\) ...'' or similar], i.e. \((\alpha_s, \alpha_{s+1})\), since this is
\((\alpha_0, \alpha_2)\) [...] It therefore contains \((\alpha_0, \alpha_2)\) since this
is \((\alpha_0, \alpha_2)\) [handwritten marginal formulas indicating
\((\alpha_0, \alpha_2)(\alpha_0, \alpha_1)(\alpha_2, \alpha_1)\), etc.].

\bigskip

(if \(T > 2\)). It contains \((\alpha_0, \alpha_3)\) which is
\((\alpha_0, \alpha_3)(\alpha_0, \alpha_1)(\alpha_2, \alpha_1)(\alpha_2, \alpha_3)\) [...]
and repeating the argument it contains \((\alpha_0, \alpha_r)\)
if \(T > 3\), and repeating the argument it contains
\((\alpha_0, \alpha_r)\) for every \(\alpha_0 < r\). Finally it contains
\((\alpha_{p1}, \alpha_{p2})\) since this is \(R^{(m^p)}(\alpha_0, \alpha_2)\, R^{(p^{-1})}\)
if \(\nu \neq pC(T)\). Thus \(J\) contains every
two-cycle (and every three-cycle).

\bigskip

\textit{Notes:}

1. Parenthetical remarks such as ``(\(\alpha_0, \alpha_2\))(\(\alpha_0, \alpha_1\))(\(\alpha_2, \alpha_1\))'' are handwritten corrections or annotations in the margins.  
2. Phrases like ``must be greater than 4'' or ``must ... comm ...'' are partial handwritten notes near the main text.  
3. Where the image is unclear or text is fully struck out, bracketed ``[...]'' or ellipses indicate uncertain or missing content.

\section{5a}

Below is a best-effort transcription of the handwritten notes. Because they are somewhat informal, spacing and line breaks are preserved where feasible. 
Square brackets \([ \ldots ]\) indicate either an illegible segment, an uncertain reading, or clarifying context. 
Superscripts, subscripts, and negative subscripts appear in math mode (e.g.\ \(\alpha_{-1}\)). 
Parentheses \((\ldots)\) are as in the original notations.

\bigskip

Case 8) \((\alpha_0 \ \alpha_1)(\alpha_2 \ \alpha_{-2})\)

b) \((\alpha_0 \ \alpha_1)(\alpha_{-1} \ \alpha_3)\)

[There is a sketch of circles labeled \(\alpha_0\), \(\alpha_1\), \(\alpha_2\), \(\alpha_3\),
 with arrows from \(\alpha_0 \to \alpha_1 \to \alpha_3\), and so on.]

Then equivalently

\((\alpha_0 \ \alpha_1)(\alpha_2 \ \alpha_{-2})\)  Then [illegible scribbled text]
\((\alpha_3 \ \alpha_{-2})(\alpha_{-1} \ \alpha_{-5}?)\)  [unclear or partially struck out]

gives  \((\alpha_0 \ \alpha_1)(\alpha_2 \ \alpha_{-3})\) if \(T > 7\)

ok

[More heavily scribbled writing follows, indecipherable.]

\bigskip

\textit{Notes:}
- The circle-and-arrow diagram appears to show some permutation cycles or orbits labeled ``\(\alpha_0\), \(\alpha_1\), \(\alpha_2\), \(\alpha_3\),'' along with possibly arrows indicating transitions.  
- Negative subscripts like ``\(\alpha_{-1}\)'' or ``\(\alpha_{-2}\)'' appear as part of Turing's notation.  
- The last lines are almost fully scribbled out, so their precise content cannot be recovered.

\section{5}

Below is a careful line-by-line rendering of both the typed text and the handwritten/marginal notes visible in your image.  
Where words or symbols are too faint or otherwise unclear, bracketed ellipses ``[...]'' or notes indicate this.  
Strikethrough text is shown with ``~~...~~,'' and handwritten insertions or comments appear in square brackets.

\bigskip

c) \((\alpha_0, \alpha_2)(\alpha_2, \alpha_3)\) belongs to \(J,\) where ~~are all different.~~ 
(\text{handwritten note near ``$\alpha_2, \alpha_3$''}:
  ``3 different ??? for $x_2, x_3$'')

d) \((\alpha_0, \alpha_1)(\beta, \beta')\) belongs to \(J,\) where ~~are all different.~~ 
([\text{handwritten note}: ``$\beta$ different ???''])

e) \((\alpha_0, \alpha_1)(\alpha_{-1}, \alpha_2)\) belongs to \(J\), \([\text{handwritten next to it}: ``(???)'']\)

f) \((\alpha_0, \alpha_1)(\alpha_2, \alpha_3)\) belongs to \(J\), \([\text{handwritten next to it}: ``(???)'']\)

\bigskip

\([P.T.O. \rightarrow]\)   [handwritten arrow meaning ``please turn over'']

\bigskip

It is easily seen th\ t cases b) and c) are essentially
(by changing th\ e sigh of \(m\)) ~~and that e) and f) are~~ 
the same. In case a) since \((\alpha, \alpha_2)(\beta', \beta')\) belongs 
~~[typed text here partially crossed out or obscured]~~

\bigskip

[Below that, handwritten permutations:]
\(\bigl((\alpha_0, \alpha_1)(\beta)\bigr)\bigl((\alpha_0, \alpha_1)(\beta')\bigr) = (\alpha_0, \alpha_1)(\beta)(\alpha_0, \alpha_1)(\beta')\)

[handwritten arrows and notes continuing the algebra, for example]
\(= (\alpha_0, \alpha_1)(\beta)(\alpha_0, \alpha_2)(\beta') \ldots\)
\(\text{does } (\alpha_0, \alpha_1)(\beta)(\alpha_0, \alpha_2)(\beta') = \ldots\)

\bigskip

\textit{Notes:}
- Lines c) and d) ended with ``are all different.'' in the typescript, struck through.  
- Marginal notes about ``3 different ??? for \(\alpha_2, \alpha_3\)'' are handwritten, only partially legible.  
- The final lines (``It is easily seen ...'') are partially crossed out, so text about ``and that e) and f) are the same'' is partly visible.  The subsequent permutations in parentheses are mostly in pen.

\section{7}

Below is a faithful transcription of the typed text in the image, including spacing and any apparent typos or irregularities (for example, ``±t'' instead of ``It''). 
Square brackets are used for clarifications where something is ambiguous:

\bigskip

\noindent
\(\pm t\) is very easy to apply theorem II. We my first express 
\(U, UR, UR^2\) etc., in cycles: this my be done for inst nce by 
writin g the alph bet out double and also writing out 
the sequence \(UA, UB, \ldots UZ\). By putting the former above the 
letter in v rious positions we get th e permut tions n \(UR^s\). 
Among these we may look for permut tions which h ave a three 
cycle end all other cycles of length pr me to 3. By r ising this 
to an ppropri te power we obt in e three cycle which my or my 
not satisfy the conditions in theorem II. If we are not 
suc cesful we may use other permut tions in \(H J\). We may also 
be able in e similer way to generate e permut tion which j 
a pair of two-cycles.

\bigskip

The following upright wes chosen at rendom :*

\bigskip

ABCDEFGHIJKLMNOPQRSTUVWXYZ

MNYTFBGRSLAKOEWKPCJQZDHVUI

\bigskip

In cycles it is \((AMOWHRCYUZISJLXVDTQPK)(BNEF)(G) = U\). 
Then \(U^{22} := (BE)(NF)\). 

\bigskip

The distance BE is 3, which is prime to 26. The dist nce NF is 
8. Hence theorem II applies, end \(J\) includesthe whole of \(A,\) 
and therefore \(H\) includes \(A\).

\section{11}

Below is a faithful transcription of the typed text, preserving spacing and typos:

\bigskip

\noindent
The detailed search

\(T = 1,2,3,4\)

\(\pm t\) is not difficult to prove that there are no exception l 
groups when \(T\) is 1,2 or 3. The case \(T=4\) needs special 
investigation as it has been expres ly excluded from theorem \(\pm I\). 
The exceptional uprights in this case are \((1), (13), (24), (13)(24),\) 
\((12)(34), (32)(14), (1234), (4321)\). The exceptional groups \(H\) 
are the identity, the cyclic groups \(((1234)), ((13)(24))\), the 
four-group, consisting of the identity and all permutations of 
form \((\alpha \ \beta)(\gamma \ \delta)\), and a group isomorphic with the four 
group and generated by \((13)\) and \((24)\).

\bigskip

\(T = 5\)

\section{13}

Below is a careful transcription of typed text, preserving spacing and typos:

\bigskip

\noindent
\(T = 8\)

This needs r-ther more investigation th-n the previous
cases, partly because it is the l-rgest numebr yet considered,
and partly because it has more factots.

Obviously the permutations which commute with \((R)\) or with
a power of \(R\) or generate an intransitive group will be
exceptional. We will consider that we are looking for other
forms of exceptional xxxx upright.

We have various means for dealing with th e permutations.

\section{14}

\noindent
This is indicated by the v lue of the commut\(\cdot\)tor and ixxx ``O.K.''
Wh en all these fail a query will be shown, end the upright
investigated further later.

\(t = 1\)

We may first go over the main plan, considering seperately
what is to be done withthe var\`{o}us classes of conjugates in
the symmetric group.

666cycles. These are left asi e till the double threes
have b\`{e}en conidered.

Double threes. Th ese pre arr nged in pairs (as transformed
by \((CH)(EF)(DG)^v\)) which leaves A,B fixed and xxxxxxxx
satisfies \(a^v R v^{-1} R'\) and dealt with in detail.

Triple twos. Very few of these n eed to be considered in
det il. Those \(\beta\)ith the pair \((CH)\) give \((BAC)/(...)\) \(\beta\)ith other cycles
on a slide, and so are either O.K.\ under d) or equivalent to
a double three. Those with the pair \((DH)\),re reduced to a
\(t=4\) case under a), and those with the pair \((GG)\) are paired with
ones having \((DH)\).

Four-and-twos need only be considered when their squares are
intransitive \(\beta\) or commute with \(R^4\) by theorem II.

Other cases consist of ones wh ree th ree or more letters [...]

\section{15a}

Below is a best-effort, verbatim transcription of a typed page diagonally crossed out. 
Spacing, punctuation, and typographical quirks are preserved, with bracketed ellipses for unclear text. 
Since the entire page is struck through, it was presumably meant to be discarded or revised:

\bigskip

th e analysis rther further. Theorem III effectively enables
us to confine our attention to sequences
where g is constant throughout each coset of C the commutator
group of H. If j represents th e function which is equal to
th e reciprocal of th e index of C, for values in C, and is x
0 outside, then \(Rj\) operting on any function converts it into
one which is constant on the cosets, and has no effect applied
to functions already having this property: in fact it averages
over cosets. This operator \(Rj\) commutes with all \(Rp\). Also if
xxx and

then
(if g constant in cosets). Thus we can xxxxxx work with \(f'\)
instead of \(f\) and confine ourselves entirely to functions
constant in the cosets, i.e.\ effectively to functions
in the factor-group \(H/C\), which is Abelian. We hve thüs
reduced the original problem to the c se of en Abelian group.

\section{16}

Below is a literal transcription, preserving spacing, punctuation, and typos:

\bigskip

\noindent
The upr\`{a}ght \((CDF)(EGH)\) is exceptionl en d the corresponding
group consisgts of the elements withth e invariants

\[
\begin{array}{ll}
11111111 & 8 \quad \text{elements} \quad 11111111 \\
12214554 & 64 \quad\qquad\qquad\quad 25527667 \\
13272515 & 64 \quad\qquad\qquad\quad 13245423 \\
15187216 & 64 \quad\qquad\qquad\quad 34657564 \\
24636425 & 64 \quad\qquad\qquad\quad 14737415 \\
33476674 & 64 \quad\qquad\qquad\quad 12216336 \\
77777777 & 8  \quad\qquad\qquad\quad 77777777 \\
         & 336
\end{array}
\]

Transformation of the grou\(\mathrm{p}\) with \((AG)(CH)(EC)\), which commutes
with \(R\), gives another group which contins \((CFG)(DEH)\).
The invariants of this latter group are given in th e
last column. These invariants are useful for verifying
that other exceptional uprights belong to these groups.

XXXXXXXXXX

We have to investig te fi\~{n}\~{n}xx the six-cycles whose squares
are exceptional. They are shown below

\((CEDGFH)\)\quad\(\;\;\;X\)
\((CGDHFE)\)\quad Invariant 15132723 (above)
\((CHDETG)\)\quad\(\;\;\;X\)
\((CDFEGH)\)\quad Invariant 12216336 (above)
\([CFFEGD?]\)\quad \(\;X\;\;X\)
\([CHE... ?]\)\quad [possibly ``Slide (ABDH)(CE) O.K.''?]
\([CDEHGF?]\)\quad \(\;X\;\;X\)
\([CFEDGH?]\)\quad \(\;X\)

[Handwritten formulas or notes faintly behind text, referencing \((CE)(ABDH)\) etc.]

\section{19a}

Below is a best-effort, line-by-line transcription of typed text also crossed out diagonally. 
Spacing, punctuation, and typographical quirks are preserved. 
Where characters or words are unclear, bracketed ellipses indicate:

\bigskip

\noindent
If
\[
r = \prod_{i=1}^{N} b_i^{-1} r_i^{s_i} b_i,
\]
then 
\[
\nu\nu(a^{-r} \, a) = \nu\nu\Bigl(\prod_{i=1}^{N} \bigl((b_i a)^{-1} \, r_i^{s_i} \, b_i a \bigr)\Bigr)
\]
i.e. (3,f\(_1\)) is satisfied. Also
\[
\chi_a\bigl(a^{-1} \, r_a \, (a \, b)\bigr)
  = \chi_a\bigl(\chi_a(\chi_a^{-1} b)\bigr)
\]
[handwritten marks here]

so that (3,f\(_1\), \(a^{-1} r_a, b\)) is satisfied. But if (3,f\(_1\), \(s_i, r_i, b\))
are satisfied for all \(b\) then (3,f\(_1\), \(r_s, b\)) is satisfied for all \(b\).
Consequently (3,f\(_1\)) is satisfied and the corollary to theorem 1 applies.

In the cases when the centre of \(\mathfrak{M}\) consists either of the
identity alone or of the whole group there is always a solution
of the equations (6). The expressions on the right h nd sides
of these equations always represent centre elements, so that
in the case where the centre consists of the identity alone,
there is a solution by putting \(\zeta_i = 1\) for each \(i\). If \(\mathfrak{M}\) is Abelian
we [text unclear or incomplete]. For the general case we have
to be able to find all the relations (7).

\section{20}

Below is a literal transcription of typed text, reproduced verbatim:

\bigskip

\noindent
\(t = 2\)

We xxxxx find it worth while to apply the principle (ii)
bn a rather larger scale. There are four permutations \(V\)
which leave \(A\) and \(C\) fixed; xxxxx they are

\bigskip

\noindent
A B C D E F G H

C B A H G F E D

F A D G B E H

A F C H E B G D

\bigskip

From a single permutation w\(\dot{\mathrm{e}}\) thus obtain as many as four
xxxx generating isomorphic groups \(J\), e.g. from \((BDEFHG)\)

\bigskip

\((BDEFHG)\)

\((BHGFDE)\)

\((FDBGHE)\)

\((FHEBDG)\)

\bigskip

These may be trnsformed into eculvlent forms, en d the
alphabetically e rliest chosen. We permit taking the reciprocal
as a form of transformation. Thus we get \((BDEFH G),(BEDEFGH);\)
\((BGDFEH)?\) \((BDGTHE)\). By these means we reduce the six-cycles
th at need be considered do wn to 18. As before we ectully
consider first their squares xxxxxx xxxxxx (double threes)
in xx the hope that they will be unexceptionl en d the
six cycle need not be specially investigated.

\section{21}

Below is a verbatim transcription of typed text, preserving spacing and typos.  
List format with each line starting with a six-cycle in parentheses:

\bigskip

\noindent
Six cycles and double-threes

\bigskip

\((BDEFHG)\) \quad Slide \((ACG)(BFDHE)\) O.K. indirect

\((BDEFFG)\) \quad SLIDE \((AB)(CEFHD)\) O.K.

\((BDEGFH)\) \quad \((BET)(DGH).\,(CFG)(EHA)=(ATH)(CBEDG)\) O.K.

\((BDGEHF)\) \quad SLIDE \((AD)(FCB)\) O.K.

\((BDBEHG)\) \quad SLIDE \((BAG)(DCEH)\) O.K.

\((BDEHGF)\) \quad Invariant 34657564, giving group \(K'\).

\((BDFEHG)\) \quad SLIDE \((BA)(DCFGH)\) O.K.

\((BDFGHE)\) \quad SLIDE \((BA)(CFD)(HEG)\) O.K.

\((BDFHGE)\) \quad SLIDE \((BAEH)(DCF)\) O.K.

\((BDFHGE)\) \quad SLIDE \((BAEH)(DCF)\) O.K.

\((BDGFEH)\) \quad Invrint 34657564, givin g group \(K'\).

\((BDGHEF)\) \quad SLIDE \((CAE)(FHDGB)\) O.K. indirect.

\((BDGHFE)\) \quad \(\{BGF\}(DHE).(CHG)(EAF)=(ABGCE)(DHF)\) O.K. indirect.

\((BDHEFG)\) \quad SLIDE \((BAFG)(DCH)\) O.K.

\((BDHEGF)\) \quad SLIDE \((AH)(BCEDF)\) O.K.

\((BDHFGE)\) \quad SLIDE \((DAEH)(FCG)\) O.K.

\((BDHGFE)\) \quad SLIDE \((AD)(HBFCE)\) O.K.

\((BDHGEF)\) \quad SLIDE \((CAE)(DHBFG)\) O.K. indirect.

\((BEDHGF)\) \quad SLIDE \((AED)(CF)\) O.K.

\bigskip

Above analyses are done on the squares of th e six cycles i.e.
on the double threes. We must now in ve'tigte the cases of
six-cycles where the double threes were exception l

\bigskip

\((BDEHGF)\) \quad SLIDE \((BAG)(FH)(CD)\) O.K.

\((BDGHEF)\) \quad SLIDE \((BAEGH)(CD)\) O.K.

\section{22}

\noindent
Below is a line-by-line transcription of the handwritten page titled ``Triple Twos,'' then ``Four and twos and fours.'' 
Where letters or symbols are partially obscured, bracketed question marks or ellipses appear.

\subsection*{Triple Twos}

\noindent
\((BF)(DA)(GEH)\)  
\quad X  
\quad Intransitive  
\quad Slide \((A F \& D+?)(B C F)\) O.K.

\bigskip

\((BF)(DH)(GEA)\)  
\quad X  
\quad Intransitive  
\quad Slide \((AH)(BCG)(Bf)\) O.K.

\bigskip

\((BG)(DA)(EFH)\)  
\quad X  
\quad Intransitive  
\quad \((AH)(BCG)(Bf)\) Slide O.K.

\bigskip

\((BG)(DF)(EAH)\)  
\quad X  
\quad Slide \((AH)(BCG)(Bf)\) O.K.

\bigskip

\((BH)(DG)(EAF)\)  
\quad X  
\quad Intransitive  
\quad \((AH)(BCG)(Bf)\) Slide O.K.

\bigskip

\((BH)(DC)(E?F)\)  
\quad X  
\quad Slide \((A?H)(BC?)(\ldots ?)\) O.K.

(... etc. ...)

\bigskip

\subsection*{Four and twos and fours}

\noindent
We can only need consider those four-and-twos that are intransitive
or commute with \(R^4\).  Transpositions listed below:

\((EBGD)(F^H)\) \quad (...)  
\((FBGD)(CE?)\) \quad O.K.  
\((EBFC)(\ldots ?)\) \quad O.K.\quad (\((ABFC)(DA)\))  O.K.  
\((EBGA)(H? \ldots )\) \quad O.K.\quad \((EBF\ldots )(CCD)\) O.K.  
\((ED?G B)(F^H?)\)(...)  
(... etc. ...)

[One marginal note: ``These are none which leave A, C fixed ... except (13)(BH+?), which is Intransitive anyway.'']

\section{27}

Below is a literal transcription of typed text and handwritten edits. 
No Unicode punctuation; Greek letters and subscripts are rendered in math mode:

\bigskip

Frequency distribution of xxxxxxxxx group elements

We now turn to a rather different topic in connection
with the use of identical drums. Even if we know th et all
permutations are possible, will they we equally frequent?
Fortunately we can answer th is in the affirmative. The
problem will be examined under slightly more general conditions.
Xxxxxxxxxxxxxxxxxxx No assumptions will be mde about the
relationship between th e generetrors \(U_1 \ldots M? U_K\), and we will not
assume th at the basic group is the symmetric group, but some
other group \(G\).

Let us xxxxxxxxxxx suppose th at we feed xxxxxxxxx
a certin frequency distribution of g'oup elements into
a wheel; how can we c'elulate the frequency distribution of the
group elements at th e output of the wheel? Let \(g(a)\) be the
proportion of the inut elements which re \(a\), and let \(F(e)\) be
the proportion of group elements effected by the wheel which are \(e\).
Then we get output \(a\) if the input is \(b\) and the wheel effects
the group element \(a b^{-1}\). The proportion of such cases
is \(f(a b^{-1})\, g(b)\), or allowing for the different values of \(b\), a
total proportion of \(\sum_b f(a b^{-1})\, g(b)\). If then we define
the operator \(R_f\) by the equation
\[
(R_f\, g)(a) \;=\; \sum_b f(a b^{-1})\, g(b),
\]
we can say that the frequency distribution for \(n\) wheels is
given by \(R_f^{\,n-1} \,f\). We wish to determine how this function behaves
with increasing \(n\).

\section{28}

Below is a line-by-line transcription of both typed text and the most legible handwritten annotations:

\bigskip

real

We consider the xxxxxx-valued functions in th e group as
forming a xxxxx Euclidean space of \(h\) dimensions wh ere \(h\)
is the order of the g roup \(H\). We may put \((g,k)\) for
\[
\frac{1}{h}\,\sum_a g(a)\,k(a)
\]
the scalar product, and \(\Vert g\Vert\)

for the distance from the origin.

We may also put \(\overline{g} = \frac{1}{h}\,\sum_a g(a)\).
Schwarz' inequality gives at once
\(\Vert g\Vert \ge \overline{g}\), and if
we suppose \(g(a) > 0\) on all \(a\), \(g(a) > 0\) some \(a\), we shll
have \(\overline{g} > 0\). We will also suppose \(f(a) > 0\) all/b, \(f(c) > 0\) some \(a\),
\(\overline{f} = (?)\). Then we hve

\textbf{Lemma (a)}

If \(\overline{f} = 1\) then \(\Vert R_{(g)} f\Vert \le \Vert f\Vert\), and equality holds
only if \(g(\ldots)/g(x)\) is independent of \(x\) for any \(g, b\)
for which \(f(a)\neq 0\) and \(f(b)\neq 0\).

First note th at
\[
\Bigl(\frac{1}{h^2}\,\sum\,g(x)\,g(c\,x)\Bigr)^2 \le \ldots
\]
[handwritten partial math]

Then
\[
\Vert R_{(g)} f\Vert^2
  = \frac{1}{h^2}\,\sum_{a,b,x} f(a\,b^{-1})\,g(b)\,f(a\,x^{-1})\,g(x)
\]
\[
= \frac{1}{h^2}\,\sum_{c,s,x} f(c)\,f(c\,u)\,g(u\,x)\,g(x)
  \quad\text{where }u := b\,x^{-1},\; c := a\,b^{-1}
\]
\[
\le \sum_{c,s} f(c)\,f(c\,u)\,\Vert g\Vert^2
\]
\[
= [\text{something}]\,\Vert g\Vert^2
\]
equality holding in the case mentioned. Ths will enable us to \ldots

[Typed lines partially crossed out or faint:
``Let us define the 'limiting distribution' \(f'\) as the
accumulation points of the sequences \(g, R_f g, R_f^2 g, f, g,\ldots\)'']

Then lemma (a) will enable us to prove
\ldots

\section{29}

Below is a careful transcription of typed text (partly crossed out) and handwritten notes:

\bigskip

\textbf{Theorem III}

The limiting distributions for \(f\) are constant throughout
the cosets of a certain self-conjugate subgroup \(H_1\) of \(H\). \(H_1\)
consists of all expressions xxxxxxxxxxxxxxxxxxxxxxx
of the form \(U_{m_1}\,U_{m_2}\,\ldots\,U_{m_p}\) where the sum \(\sum m_i = 0\). The factor
group \(H / H_1\) is cyclic. In the case each [typed text unclear]
\(g\) is \(f\) the limiting distributions have the value \(\theta\) except
in one coset of \(H_1\).

Let \(k\) be a limiting distribution. Let it be the limit of
the sequence \(R_f^r g_s, R_f^r g_j, R_f^r g\ldots\)
then [some text about norms] \(\Vert R_f^{\,n+1} g\Vert / \Vert R_f^{\,n} g\Vert > \ldots\)
Now \(\Vert R_f^{\,n} g\Vert / \Vert k\Vert \to \ldots\)
tends to the limit 1 as \(n\) tends to infinity, and therefore
\(\Vert R_f^{\,n+1} g\Vert / \Vert R_f^{\,n} g\Vert\) tends to 1. But \(\Vert R_f\,u\Vert / \Vert u\Vert\)
is a continuous function of \(u\) and therefore xxxxxxxxxxxxxxxx
limit of the sequence \(\ldots \Vert R_f^k / \Vert k\Vert = 1\)

Applying lemma (a\(\Phi\)) to this we see th t there is a function
\(\phi_\ell(4)\), defined for all expressions of form \(a\,b^{-1}\) where \(f(a)\neq 0\)
\(f(b)\neq 0\) such th t \(\bigl(\phi(y\,x) : \phi(y)(k\,x)\bigr) = \ldots\).
By applying the same argument with \(R_f^n\) in place of \(f\) we find that
there is a function \(\phi_n(4)\) defined for all expressions
of form \(a_1\,a_2\,\ldots\,a_k\,b^{-1}\,b_k^{-1}\ldots b_1^{-1}\) where \(f(a_1)f(a_2)\ldots f(h_k) \neq 0\),
such that \(\phi_n(xy) = \phi_n(x)\phi_n(y)\).
Whenever \(\phi_n(4)\) is defined. The various functions \(\phi_n(4)\)
must agree wherever their domains overlap, and
they may therefore be all represented by one symbol \(\phi\). In
fact we may say that \(\phi(x)\) is defined and has the value \(\times\)
whenever \(g(x):= \alpha \ldots\). It now appears that the domain
of definition of \(\phi(g)\) is a group, for if [handwritten: ``\(\phi(4,x) = \alpha, \phi(x)=\ldots\)'' etc.]

\section{30}

\noindent
and \(k(u_2, x) \ge k(x)\) all \(x\), then \(k(u_1, u_2 x?): \phi(u_1)k(u_2 x?): \phi(u_1), \phi(u_2), \phi(k(x))\)
all \(x\). Thus if \(y_1\) and \(y_2\) belong to the domain of
definition of \(\phi\) so does \(y_1 y_2\) and \(\phi(u_1, y_1, y_2): \phi(u_1, y_2)(\phi(y_2?))\). It
is now immediately seen that the domain of definition is \(H_1\).

The function \(\phi\) is a one dimensional representation of \(H_1\),
but it is real and positive and therefore has the value 1
last throughout \(H_1\). This argument may also be expressed without the
use of representation theory thus. Since \(H_1\) is finite any
element \(y\) of it satisfies an equation \(y^m = 1,\) therefore
~~((\(\phi(4)\))\(^m \cdot \phi(4^m)=1\) isn't non sen e   non . negativity \(\phi(y)>0\), so \(\phi(4)>1\)~~
But since \(g(x)\) is always ~~positive~~ [handwritten: ``non-negative''] ~~and so~~
this implies that \(g(x)\) is constant throughout each coset of \(H\).

It now only remains to investigate the c\(\sim\)her\(\sim\)racter of the
group \(H_1\). It is easily seen to be self-invariant or self-conjugate, 
since if \(a b a^{-1}\) belongs to \(H_1,\) and \(b\in H\), 
the total of exponents of group generators \(U_r\) in \(a b a^{-1}\) must be 0, 
those in \(a^{-1}\) canceling with those in \(a\). 
Hence \(a b a^{-1}\) belongs to \(H_1\).

Now let us take a particular generator \(U_1\). Then the cosets
\(H_1 U_1^m\) exhaust the group \(H\). For if \(p\) is an element of \(H\) it will
be a product of generators; let the total of exponents be \(m\).
Then \(U_1^{-m}\) has total exponents 0, en d so belongs to \(H_1\). 
i.e. \([a \text{ belongs to } H_1 U_1^m.]\)

If \(U_1^s\) is the lowest power of \(U_1\) which
belongs to \(H_1\) then \(H/H_1\) is evidently isomorphic with the
cyclic group of order \(s\).

In the c se th t \(g\) is \(f,\) all the
group elements for which \(R_f^{(n-1)} xxxxx\) is not zero are products
of \(n\) generators en d th erefore belong to \(H_1 U_1^n.\)

\section{31}

\noindent
\textbf{Example}

As an example let us consider the quaternion group
consisting of \(1, i, j, k, i', j', k'\), with the table

\[
\begin{array}{c|ccccccc}
   & i & j & k & i' & j' & k' & \dots \\
\hline
i' & k' & j' & i  & k  & j  & \dots & \\
k' & j  & i' & \dots &   &    &       & \\
j' & i  & k' & \dots &   &    &       & \\
\vdots &  &   &      &   &    &       & 
\end{array}
\]
(and so on, rows/columns for \(1,i,j,k,i',j',k'\)\, ...)

and let \(U_1\) be \(i\) and \(U_2\) be \(j\). The various functions \(R_f^n f\)
are given in the table below

\[
\begin{array}{c|cccccc}
  n & i & i' & j & j' & k & k' \\
\hline
 1 & \tfrac{1}{2} & \tfrac{1}{2} & 0 & 0 & 0 & 0 \\
 2 & 0 & 0 & \tfrac{1}{4} & \tfrac{1}{2} & 0 & \tfrac{1}{4} \\
 3 & 0 & \tfrac{1}{4} & 0 & \tfrac{1}{4} & 0 & \dots \\
 4 & \dots & & & & & \\
 5 & \dots & & & & & \\
 6 & \dots & & & & & \\
\end{array}
\]

It is seen th t the group \(H_1\) is the group generated by \(k\),
it has factor group which is cyclic of order 2.

\section{32a}

Below is a faithful attempt to capture the clearly legible text. 
Most typed text from the reverse side is too faint:

\bigskip

\noindent
\textbf{[Handwritten in green ink]:} On Permutation Groups

\bigskip

\noindent
\textbf{[Typed text in background is largely illegible. No further text discernible.]}

\section{32-Permutation-Groups}

Below is a literal transcription of typed text, preserving minor misspellings and strikethroughs:

\bigskip

\noindent
Case of sym metric and alternating groups

In the case under consider tion at th e beginning of our
analy sis, \(H\) was ~~unless the upright \(U\) was ex ceptl~~ eith er the symmetric or the
alternating group. In this case \(H_1\) is xxxxxxx also either
the symmetric or th e altern ting group, for it is self
conjugate in \(H\). It will b e the xxxxxxx group if the
generators are all of th e s me r rity, and the symmetric group
otherwise.

We th erefore conclude th t when th e uprigh t is not
exception al th e distributions with large numbers of
wheels are uniform throughout the alternating group (even perms).
If odd permutations are possible with the given uprigh t and
number of wheels th e distribution is uniform th roughout
th e symmetric group (all perms).

\bigskip

\end{document}
